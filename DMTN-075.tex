\documentclass[DM,authoryear,toc]{lsstdoc}
% lsstdoc documentation: https://lsst-texmf.lsst.io/lsstdoc.html

% Package imports go here.
\usepackage{bm}
\usepackage{color}

% Local commands go here.
\newcommand{\todo}[1]{{\color{red}[#1]}}

% To add a short-form title:
% \title[Short title]{Title}
\title{Cell-Based Coaddition}

% Optional subtitle
% \setDocSubtitle{A subtitle}

\author{%
Jim Bosch
}

\setDocRef{DMTN-075}

\date{\today}

% Optional: name of the document's curator
% \setDocCurator{The Curator of this Document}

\setDocAbstract{%
Design sketch and mathematics for a new approach to building coadds and constraining their inputs.
}

% Change history defined here.
% Order: oldest first.
% Fields: VERSION, DATE, DESCRIPTION, OWNER NAME.
% See LPM-51 for version number policy.
\setDocChangeRecord{%
  \addtohist{1}{YYY-MM-DD}{Unreleased.}{Jim Bosch}
}

\begin{document}

% Create the title page.
% Table of contents is added automatically with the "toc" class option.
\maketitle

\section{Optimal Coadds}

All quantities are defined within a single cell, which is a rectangle with integer-pixel boundaries in the coadd coordinate systen.
Input-image quantities in cells enclose a larger area: they include any pixel that could contribute to that cell on the coadd.
This means that some input-image pixels contribute to multiple cells.
We assume that cells are sufficiently small that all PSFs, coordinate transforms, and backgrounds can be considered constant over any \emph{two} adjacent cells.

\subsection{Well-Sampled Data}

Definitions:
\begin{itemize}
\item $\bm{\mu}$, $\bm{\nu}$: vector positions in coadd coordinates (each of $\bm{\mu}$ and $\bm{\nu}$ is a tuple of two coordinates).
\item $\lambda$: an \emph{index} over wavelength bins
\item $\bm{A}_i$: affine mapping from input image $i$ coordinates $\bm{x}_i$to coadd coordinates.
\item $\bm{x}_{i}$, $\bm{y}_i$: positions in the coordinate system of input image $i$ (each of $\bm{x}$ and $\bm{y}$ is a tuple of two coordinates).
\item $z_{i}[\bm{x}_i]$: the value of the pixel at $\bm{x}_i$ in input image $i$, including backgrounds.
\item $\phi_{i,\lambda}(\bm{x}-\bm{y})$: PSF that maps true-sky flux at $\bm{x}$ to observed flux in input image $i$ at $\Delta\bm{y}$, integrated over wavelength bin $\lambda$.  Constrained such that
$\int \phi_{i,\lambda}(\bm{x}) d^2\bm{x} = 1$ for all $\lambda$.
\item $\tau_{i,\lambda}$: Photometric transmission that maps true-sky flux to observed flux in input image $i$, integrated over wavelength bin $\lambda$
\item $C_{i}[\bm{x}_i, \bm{y}_i]$: noise covariance between pixels $\bm{x}_i$ and $\bm{y}_i$ on input image $i$.  Assumed Gaussian and uncorrelated with neighboring pixels.
\item $N$: the number of input images.
\item $W$: the number of wavelength bins.
\item $\bm{E}_i$: the set of all pixel indices in input image $i$ that contribute to the current cell.
\item $\bm{F}$: the set of all coadd pixel indices in the cell.
\item $\alpha_{\lambda}(\bm{\mu})$: true above-the-atmosphere scene in coadd coordinates, integrated over wavelength bin $\lambda$, not including backgrounds.
\item $\beta_{i,\lambda}$: sky and other backgrounds in input image $i$ and wavelength bin $\lambda$.
\end{itemize}

We begin by using sinc interpolation to resample each input image to the input coordinate system:
\begin{align}
    \hat{z}_{i}[\bm{\mu}] =
        \sum_{\bm{x}_i \in \bm{E}_i}
            \mathrm{sinc}(\pi(\bm{A}_i^{-1}\bm{\mu} - \bm{x}_i))
            \, z_{i}[\bm{x}_i]
\end{align}
with uncertainty
\begin{align}
    \hat{C}_{i}[\bm{\mu}, \bm{\nu}] =
        \sum_{\bm{x}_i \in \bm{E}_i}
            \mathrm{sinc}(\pi(\bm{A}_i^{-1}\bm{\mu} - \bm{x}_i))
            \,\mathrm{sinc}(\pi(\bm{A}_i^{-1}\bm{\nu} - \bm{y}_i))
            \, C_{i}[\bm{x}_i, \bm{y}_i]
\end{align}
and PSF
\begin{align}
    \hat{\phi}_{i,\lambda}(\bm{\mu}-\bm{\nu}) =
        \frac{1}{|A_i|}
        \phi_{i,\lambda}(\bm{A}_i^{-1}\bm{\mu} - \bm{A}_i^{-1}\bm{\nu})
\end{align}
The photometric calibration $\tau_{i,\lambda}$ and background $\beta_{i,\lambda}$ are unchanged under this transformation \todo{check this!}.

The joint log-likelihood for all input images is then:
\begin{align}
L \equiv &
    \frac{1}{2} \sum_{i=0}^{N}
        \sum_{\bm{\mu} \in \bm{F}}
        \sum_{\bm{\nu} \in \bm{F}}
        \left[
        \hat{C}_i^{-1}[\bm{\mu}, \bm{\nu}]
        \right.
        \\
    & \quad \times
        \left(
            \hat{z}_i[\bm{\mu}]
            - \sum_{\lambda=0}^{\lambda < W} \beta_{i,\lambda}
            - \sum_{\lambda=0}^{\lambda < W} \tau_{i,\lambda}
                \int\!
                \, \hat{\phi}_{i,\lambda}(\bm{\mu} - \bm{\eta})
                \, \alpha_{\lambda}(\bm{\eta})
                \, d^2\bm{\eta}
        \right) \\
    & \quad \times
    \left.
        \left(
            \hat{z}_i[\bm{\nu}]
            - \sum_{\lambda=0}^{\lambda < W} \beta_{i,\lambda}
            - \sum_{\lambda=0}^{\lambda < W} \tau_{i,\lambda}
                \int\!
                \, \hat{\phi}_{i,\lambda}(\bm{\nu} - \bm{\zeta})
                \, \alpha_{\lambda}(\bm{\zeta})
                \, d^2\bm{\zeta}
        \right)
    \right]
\end{align}

While the true sky contains power at arbitrarily small spatial scales, the PSFs do not; we can always choose the coadd coordinate system such that all
input PSFs are well-sampled on that grid.
We can thus exactly represent the continuous PSF as a sinc interpolation of
discrete samples from it:
\begin{align}
    \hat{\phi}_{i,\lambda}(\bm{\mu}-\bm{\eta}) =
    \sum_{\bm{\gamma} \in \bm{P}}
        \mathrm{sinc}(\pi(\bm{\eta}-\bm{\gamma}))
        \,\hat{\phi}_{i,\lambda}[\bm{\mu}-\bm{\gamma}]
\end{align}
We can then insert this into the convolution integral,
\begin{align}
    \int\!
        \, \hat{\phi}_{i,\lambda}(\bm{\mu}-\bm{\eta})
        \, \alpha_{\lambda}(\bm{\eta})
        \, d^2\bm{\eta}
    = &
    \sum_{\bm{\gamma} \in \bm{P}}
        \,\hat{\phi}_{i,\lambda}[\bm{\mu}-\bm{\gamma}]
        \int\!
        \mathrm{sinc}(\pi(\bm{\eta}-\bm{\gamma}))
        \, \alpha_{\lambda}(\bm{\eta})
        \, d^2\bm{\eta}
\end{align}

% Include all the relevant bib files.
% https://lsst-texmf.lsst.io/lsstdoc.html#bibliographies
\bibliography{lsst,lsst-dm,refs_ads,refs,books}

\end{document}
