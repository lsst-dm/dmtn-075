

\documentclass[DM,authoryear,toc]{lsstdoc}
% lsstdoc documentation: https://lsst-texmf.lsst.io/lsstdoc.html

% Package imports go here.
\usepackage{bm}

% Local commands go here.

% To add a short-form title:
% \title[Short title]{Title}
\title{Cell-Based Coaddition}

% Optional subtitle
% \setDocSubtitle{A subtitle}

\author{%
Jim Bosch
}

\setDocRef{DMTN-075}

\date{\today}

% Optional: name of the document's curator
% \setDocCurator{The Curator of this Document}

\setDocAbstract{%
Design sketch and mathematics for a new approach to building coadds and constraining their inputs.
}

% Change history defined here.
% Order: oldest first.
% Fields: VERSION, DATE, DESCRIPTION, OWNER NAME.
% See LPM-51 for version number policy.
\setDocChangeRecord{%
  \addtohist{1}{YYY-MM-DD}{Unreleased.}{Jim Bosch}
}

\begin{document}

% Create the title page.
% Table of contents is added automatically with the "toc" class option.
\maketitle

\section{Mathematics}

All quantities are defined within a single cell, which is a rectangle with integer-pixel boundaries in the coordinate systen.
We assume that cells are sufficiently small that all PSFs, coordinate transforms, and backgrounds can be considered constant over any \emph{two} adjacent cells.

Definitions:

\begin{itemize}
\item $\bm{\mu}$, $\bm{\nu}$: vector positions in coadd coordinates (each of $\bm{\mu}$ and $\bm{\nu}$ is a tuple of two coordinates)
\item $\lambda$: an \emph{index} over wavelength bins
\item $\alpha_{\lambda}(\bm{\mu})$: true above-the-atmosphere scene in coadd coordinates, integrated over wavelength bin $\lambda$, not including backgrounds.
\item $\beta_{i,\lambda}$: sky and other backgrounds in input image $i$ and wavelength bin $\lambda$
\item $\bm{A}_i$: affine mapping from input image $i$ coordinates $\bm{x}_i$to coadd coordinates.
\item $\bm{x}_{i}$, $\bm{y}_i$: positions in the coordinate system of input image $i$ (each of $\bm{x}$ and $\bm{y}$ is a tuple of two coordinates).
\item $\phi_{i,\lambda}(\bm{x})$: PSF that maps true-sky flux to observed flux in input image $i$ a distance $\bm{x}$ away (in input image pixel coordinates), integrated over wavelength bin $\lambda$.  Constrained such that
$\int \phi_{i,\lambda}(\bm{x}) d^2\bm{x} = 1$ for all $\lambda$.
\item $\tau_{i,\lambda}$: Photometric transmission that maps true-sky flux to observed flux in input image $i$, integrated over wavelength bin $\lambda$
\item $z_{i}[\bm{x}]$: the value of the pixel at $\bm{x}$ in input image $i$, including backgrounds.
\item $\sigma_{i,j}$: RMS uncertainty of pixel $j$ on input image $i$.  Assumed Gaussian and uncorrelated with neighboring pixels.
\item $N$: the number of input images.
\item $\bm{E}_i$: the set of all pixel indices in input image $i$.
\item $W$: the number of wavelength bins.
\end{itemize}

The log-likelihood for all input images is:
\begin{align}
L \equiv &
    \frac{1}{2} \sum_{i=0}^{i<N} \sum_{\bm{x} \in \bm{E}_i}
        \frac{1}{\sigma_{i}^2[\bm{x}_i]}
        \left[
            z_{i}[\bm{x}_i]
            - \sum_{\lambda=0}^{\lambda < W} \beta_{i,\lambda}
            - \sum_{\lambda=0}^{\lambda < W} \tau_{i,\lambda}
                \int\!
                \, \phi_{i,\lambda}(\bm{x}_i - \bm{y}_i)
                \, \alpha_{\lambda}(\bm{A}_i\bm{y}_i)
                \, d^2\bm{y}_i
        \right]^2 \\
    = &
    \frac{1}{2} \sum_{i=0}^{i<N} \sum_{\bm{x} \in \bm{E}_i}
        \frac{1}{\sigma_{i}^2[\bm{x}_i]}
        \left[
            z_{i}[\bm{x}_i]
            - \sum_{\lambda=0}^{\lambda < W} \beta_{i,\lambda}
            - \sum_{\lambda=0}^{\lambda < W} \tau_{i,\lambda}
                \int\!
                \, \phi_{i,\lambda}(
                    \bm{A}_i^{-1}[\bm{A}_i\bm{x}_i - \bm{\mu}]
                )
                \, \alpha_{\lambda}(\bm{\mu})
                \, \frac{d^2\bm{\mu}}{|A_i|}
        \right]^2
\end{align}

Regardless of whether the PSF of any particular image is well-sampled on that image's own grid, we can define the coadd coordinate system such that all input PSFs are well-sampled on its grid.
This means we can exactly represent the PSF of an input image as a sinc-weighted sum over discrete evaluations:
\begin{align}
    \phi_{i,\lambda}(\bm{A}_i^{-1}\bm{\mu})
    = &
     \sum_{\bm{\nu} \in \bm{P}}
        \phi_{i,\lambda}(\bm{A}_i^{-1}\bm{\nu})
        \, \mathrm{sinc}(\pi[\bm{\mu}-\bm{\nu}])
\end{align}
Note that this is a 2-d sum over a 2-d sinc function.
The limits $\bm{P}$ enclose the region where the transformed PSF is nonzero.
Plugging this into the likelihood and rearranging,
\begin{align}
L = &
    \frac{1}{2} \sum_{i=0}^{i<N} \sum_{\bm{x} \in \bm{E}_i}
        \frac{1}{\sigma_{i}^2[\bm{x}_i]}
        \left[
            z_{i}[\bm{x}_i]
            - \sum_{\lambda=0}^{\lambda < W} \beta_{i,\lambda}
            - \sum_{\lambda=0}^{\lambda < W} \tau_{i,\lambda}
                \sum_{\bm{\nu} \in \bm{P}}
                    \phi_{i,\lambda}(\bm{A}_i^{-1}\bm{\nu})
                \int\!
                    \, \mathrm{sinc}(\pi[\bm{A}_i\bm{x}_i-\bm{\mu}-\bm{\nu}])
                \, \alpha_{\lambda}(\bm{\mu})
                \, \frac{d^2\bm{\mu}}{|A_i|}
        \right]^2
\end{align}

% Include all the relevant bib files.
% https://lsst-texmf.lsst.io/lsstdoc.html#bibliographies
\bibliography{lsst,lsst-dm,refs_ads,refs,books}

\end{document}
